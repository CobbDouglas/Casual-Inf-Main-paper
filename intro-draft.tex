% Options for packages loaded elsewhere
\PassOptionsToPackage{unicode}{hyperref}
\PassOptionsToPackage{hyphens}{url}
%
\documentclass[
  12pt,
]{article}
\usepackage{amsmath,amssymb}
\usepackage{iftex}
\ifPDFTeX
  \usepackage[T1]{fontenc}
  \usepackage[utf8]{inputenc}
  \usepackage{textcomp} % provide euro and other symbols
\else % if luatex or xetex
  \usepackage{unicode-math} % this also loads fontspec
  \defaultfontfeatures{Scale=MatchLowercase}
  \defaultfontfeatures[\rmfamily]{Ligatures=TeX,Scale=1}
\fi
\usepackage{lmodern}
\ifPDFTeX\else
  % xetex/luatex font selection
\fi
% Use upquote if available, for straight quotes in verbatim environments
\IfFileExists{upquote.sty}{\usepackage{upquote}}{}
\IfFileExists{microtype.sty}{% use microtype if available
  \usepackage[]{microtype}
  \UseMicrotypeSet[protrusion]{basicmath} % disable protrusion for tt fonts
}{}
\makeatletter
\@ifundefined{KOMAClassName}{% if non-KOMA class
  \IfFileExists{parskip.sty}{%
    \usepackage{parskip}
  }{% else
    \setlength{\parindent}{0pt}
    \setlength{\parskip}{6pt plus 2pt minus 1pt}}
}{% if KOMA class
  \KOMAoptions{parskip=half}}
\makeatother
\usepackage{xcolor}
\usepackage[margin=1in]{geometry}
\usepackage{graphicx}
\makeatletter
\def\maxwidth{\ifdim\Gin@nat@width>\linewidth\linewidth\else\Gin@nat@width\fi}
\def\maxheight{\ifdim\Gin@nat@height>\textheight\textheight\else\Gin@nat@height\fi}
\makeatother
% Scale images if necessary, so that they will not overflow the page
% margins by default, and it is still possible to overwrite the defaults
% using explicit options in \includegraphics[width, height, ...]{}
\setkeys{Gin}{width=\maxwidth,height=\maxheight,keepaspectratio}
% Set default figure placement to htbp
\makeatletter
\def\fps@figure{htbp}
\makeatother
\setlength{\emergencystretch}{3em} % prevent overfull lines
\providecommand{\tightlist}{%
  \setlength{\itemsep}{0pt}\setlength{\parskip}{0pt}}
\setcounter{secnumdepth}{-\maxdimen} % remove section numbering
\newlength{\cslhangindent}
\setlength{\cslhangindent}{1.5em}
\newlength{\csllabelwidth}
\setlength{\csllabelwidth}{3em}
\newlength{\cslentryspacingunit} % times entry-spacing
\setlength{\cslentryspacingunit}{\parskip}
\newenvironment{CSLReferences}[2] % #1 hanging-ident, #2 entry spacing
 {% don't indent paragraphs
  \setlength{\parindent}{0pt}
  % turn on hanging indent if param 1 is 1
  \ifodd #1
  \let\oldpar\par
  \def\par{\hangindent=\cslhangindent\oldpar}
  \fi
  % set entry spacing
  \setlength{\parskip}{#2\cslentryspacingunit}
 }%
 {}
\usepackage{calc}
\newcommand{\CSLBlock}[1]{#1\hfill\break}
\newcommand{\CSLLeftMargin}[1]{\parbox[t]{\csllabelwidth}{#1}}
\newcommand{\CSLRightInline}[1]{\parbox[t]{\linewidth - \csllabelwidth}{#1}\break}
\newcommand{\CSLIndent}[1]{\hspace{\cslhangindent}#1}
\usepackage{setspace}\doublespacing
\ifLuaTeX
  \usepackage{selnolig}  % disable illegal ligatures
\fi
\IfFileExists{bookmark.sty}{\usepackage{bookmark}}{\usepackage{hyperref}}
\IfFileExists{xurl.sty}{\usepackage{xurl}}{} % add URL line breaks if available
\urlstyle{same}
\hypersetup{
  pdftitle={Insurance Parity Laws and Reducing Suicides: Replicating Lang (2013)},
  hidelinks,
  pdfcreator={LaTeX via pandoc}}

\title{Insurance Parity Laws and Reducing Suicides: Replicating Lang
(2013)}
\usepackage{etoolbox}
\makeatletter
\providecommand{\subtitle}[1]{% add subtitle to \maketitle
  \apptocmd{\@title}{\par {\large #1 \par}}{}{}
}
\makeatother
\subtitle{Daniel Ownby}
\author{}
\date{\vspace{-2.5em}02/01/2024}

\begin{document}
\maketitle
\begin{abstract}
Over the past two decades, despite efforts from psychologists and mental
health professionals alike, the United States has seen a steady growth
of suicides despite nearly all other Western countries experiencing the
contrary. Apart of this growth could be attributed to the lagging
implementation of mental health care pairty. The Mental Health Parity
Act of 1996 (MHPA) was critized during the turn of the century for not
doing enough to restrict insurance companies ability to discrimate
between treating bodily and mental injury. In this study, I attempt to
replicate and build upon results from Lang (2013) using additional data,
propensity score matching methods along with heterogenous difference in
difference.
\end{abstract}

\newpage

\hypertarget{introduction}{%
\subsection{Introduction}\label{introduction}}

( DELETE-\/-\/-\/- 1.Hook,2. ResearchQ 3.Antecedents,4.value added,
5.Roadmap)

According to 2019 Survey for Drug use and Health, it was estimated that
at least 51.5 million adults in the United States had some sort of
mental illness. In the same year, 13.1 million were estimated to have a
serious mental illness that resulted in serious functional impairment or
interferes with at least one or more major life activity. Only 65.5\% of
those 13.1 million received any sort of mental health treatment in the
past year. According to the Center for Disease Control's WISQARS Leading
Causes of Death Report, Suicides are the second leading cause of death
amongst people aged 10-34 and the fourth from 35-44 in the United
States. Suicide rates have gradually increased over the past two
decades, starting with 10.5 per 100,000 people to 14.2 per 10,000 in
2018. Suicide rates vary from state to state with both east and west
coasts supporting low rates such as 7.4 per 100,000 while mid-western
states suffer from rates as high as 25 per 100,000. Several Sources
outline the negative effects not only through statistical life
projections and productivity losses but more generally how devastating
the preventable loss of life has on communities. (Klick and Markowitz
2006; Lang 2013)

The Federal Mental Health Parity Act of 1996 prevented group health plan
and insurance issuers from offering less mental health or substance
abuse coverage benefits compared to regular medical coverage. If a
provider gave mental health services, they couldn't offer benefit
limitations that they wouldn't otherwise give to their same
medical/surgical coverage. Most states by 2002 instated mental health
parity laws alongside further stipulations with varying degrees of
restrictiveness and exemptions.

\hypertarget{lang-2013}{%
\subsubsection{Lang (2013)}\label{lang-2013}}

My seminal paper, Lang (2013) attempts to identify causal effects using
difference-in-difference methods and fixed effects using two policy
shocks, the aftereffects of the Federal Health Parity Act of 1996 and
The Affordable Care Act of 2008. Lang (2013) showed a statistically
significant effect of a 4-7\% decrease in suicide rate after policy
implementation. I draw the same data detailed in the study but add years
spanning from 1990 to 2016, 36 years of data in total. I run my
difference in difference using regression unlike Lang (2013) addition to
conduct propensity score matching methods to achieve a better balance
between covariates between control and treatment states.

\hypertarget{parity-laws}{%
\subsubsection{Parity laws:}\label{parity-laws}}

~Any state implementing a law that requires insurance packages to
include access to mental health services and to have those services at
parity with any other physical service is flagged as a parity state.
This type of law is the strongest type amongst the ones implemented and
is the type expected to create an effect this study investigates. A less
strict version of the parity law is the ``mandated offering'' law, which
does not force insurance package providers to provide mental health
services in the first place. This can be a crucial difference when it
comes to further analysis but for the purposes of this study both are
lumped together as a Parity state.

\hypertarget{literature}{%
\subsection{Literature}\label{literature}}

\hypertarget{data}{%
\subsection{Data}\label{data}}

\hypertarget{methods}{%
\subsection{Methods}\label{methods}}

\hypertarget{results}{%
\subsection{Results}\label{results}}

\hypertarget{discussion}{%
\subsection{Discussion}\label{discussion}}

\hypertarget{conclusion}{%
\subsection*{Conclusion}\label{conclusion}}
\addcontentsline{toc}{subsection}{Conclusion}

\hypertarget{refs}{}
\begin{CSLReferences}{1}{0}
\leavevmode\vadjust pre{\hypertarget{ref-klick2006}{}}%
Klick, Jonathan, and Sara Markowitz. 2006. {``Are Mental Health
Insurance Mandates Effective? Evidence from Suicides.''} \emph{Health
Economics} 15 (1): 83--97. \url{https://doi.org/10.1002/hec.1023}.

\leavevmode\vadjust pre{\hypertarget{ref-lang2013}{}}%
Lang, Matthew. 2013. {``The Impact of Mental Health Insurance Laws on
State Suicide Rates.''} \emph{Health Economics} 22 (1): 73--88.
\url{https://doi.org/10.1002/hec.1816}.

\end{CSLReferences}

\end{document}
